\section{Introduction (approx. 5 pages)} \label{sec:intro}
\subsection{Motivation}
With the rise of modern-day technology, privacy and data security are more critical than ever. Cryptographic schemes based on conjectured hard mathematical problems lie as a key building block for enabling such properties on modern technologies. In 1994, an algorithm, due to Peter Shor \cite{Shor_1997}, was proposed that would enable computationally efficient attacks on many currently in-use cryptographic schemes using sufficiently large quantum computers. Although Shor's algorithm posed a threat to cryptographic schemes based on number theoretic problems, like RSA and DSA, an actual sufficiently-sized quantum computer was yet to be seen. 

With an eye on what a computing device based on quantum mechanics could further provide modern societies, researchers worldwide started working on practical implementations of such computing devices. The size of such quantum computers is popularly determined by the amount of \textit{quantum bits}, also denoted \textit{qubits}, present in the system. The amount of qubits present in quantum computer systems has drastically increased over the years, with IBM having tripled the number of qubits from 2021 to 2022 in its Osprey quantum processor \cite{IBM_Osprey}. This indicates rapid growth and a rapidly increasing threat against the trusted number-theoretic cryptosystems Shor challenged, albeit theoretically, in 1997.

With the rising interest in practical implementations of quantum computers, the National Institute of Technology and Standards (NIST, USA) called for new quantum-resistant cryptographic schemes, also called \textit{post-quantum cryptography}, to be standardized. This process was announced in 2016 and has up to this point resulted in four submission rounds. In 2022, NIST ended the third submission round by selecting the CRYSTALS-KYBER scheme as a new standard for cryptographic key-establishment algorithms as well as FALCON, CRYSTALS-Dilithium and SPHINCS+ as new standards for digital signatures. Even though four new schemes were selected for standardization, NIST called for a fourth submission round, asking for cryptographic schemes not based on lattice problems \cite{nist_third_rd_update}.

One family of post-quantum cryptosystems is that of multivariate cryptography. This branch of cryptography revolves around two mathematical problems, the MQ problem, and the IP problem. The central problem is that of the MQ problem \cite{Ding2020}.

\begin{defn}[The MQ Problem]\label{sec1:def:mq}
Given a system of $m$ multivariate quadratic polynomials $\mathcal{P} = \{p^{(1)}, \dots, p^{(m)}\}$, over the ring of polynomials in $n$ variables $\eff[x_1, \dots, x_n]$, find values $\Bar{\mathbf{x}} = (\Bar{x}_1, \dots, \Bar{x}_n)$ that satisfy
$$
    p^{(1)}(\mathbf{\Bar{\mathbf{x}}}) =  p^{(2)}(\Bar{\mathbf{x}}) = \dots = p^{(m)}(\Bar{\mathbf{x}}) = 0 
$$
Here, $\eff = \eff_q$ is a finite field of $q$ elements. \td{PUT IN PREREQUISITES?}
\end{defn}

Choosing quadratic polynomials over other degrees of polynomials is often a product of efficiency.

Given the reliance on the MQ problem in post-quantum cryptography, having more \textit{efficient} solvers for these systems (MQ solvers) will directly impact the security of multivariate cryptographic schemes. Using existing and well-known solvers, the NIST candidate scheme \texttt{Rainbow} was broken by Ward Beullens in \cite{cryptoeprint:2022/214}, which shows the necessity for these types of solvers in the cryptanalysis and parameter choice of post quantum cryptographic schemes.

However, the goal of more efficient \textit{MQ-solvers} is not only important for post-quantum cryptography. One umbrella term for different attacks on general cryptographic schemes is that of \textit{algebraic cryptanalysis}. The goal with algebraic cryptanalysis is to map the cipher in question to a system of polynomials (alongside any further necessary information) which, when solved, yields the secret key (in this case for a symmetric key cipher).

Some examples of a reduction from breaking a cipher to the MQ problem are those of \cite{nover2005algebraic}. The first attack, due to Courtois and Pierprzyk \cite{courtois2002cryptanalysis}, models the AES (Rijndael) cipher as a system of multivariate quadratic polynomials over the ring of integers modulo two ($\mathbb{Z}_2$). The second attack, due to Murphy and Robshaw \cite{10.1007/3-540-45708-9_1}, uses the idea of creating a new cipher called \texttt{BES} over an extension field of $\mathbb{Z}_2$. These two attacks do not successfully break AES, however, do potentially yield key extraction methods that are faster than a simple bruteforce procedure.

Other examples of algebraic cryptanalysis are general attacks like \textit{cube attacks} \cite{Videau2011} and importantly also attacks on cryptographic hash functions like that of \cite{10.1007/978-3-642-21702-9_6}. The use-cases for efficient MQ-solvers are therefore vast and an important tool in the cryptanalysis of not only PQC schemes, but also ordinary symmetric ciphers and cryptographic hash functions. \td{Possibly reformulate this and "ordinary ciphers".}

\subsection{Alternative methods for solving multivariate systems over $\eff_2$}

\newpage