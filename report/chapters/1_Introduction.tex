\section{Introduction} \label{sec:intro}
\subsection{Motivation}
With the rise of modern-day technology, privacy and data security are more critical than ever. Cryptographic schemes based on conjectured hard mathematical problems is a key building block in enabling such properties for modern technologies. In \cite{Shor_1997}, Peter Shor proposed an algorithm that would enable computationally efficient attacks on many currently in-use cryptographic schemes using sufficiently large quantum computers. Although Shor's algorithm posed a threat to cryptographic schemes based on number theoretic problems, like RSA and DSA, an actual sufficiently-sized quantum computer was yet to be seen. 

Now, quantum computers had already been shown to theoretically be useful in non-cryptographic research areas as well, like modelling quantum properties of particles in material sciences. Therefore, researchers worldwide started working on practical implementations of such computing devices. The size of such quantum computers is popularly determined by the amount of \textit{quantum bits}, also denoted \textit{qubits}, present in the system. The amount of qubits present in quantum computer systems has drastically increased over the years, with IBM having tripled the number of qubits from 2021 to 2022 in its Osprey quantum processor \cite{IBM_Osprey}. This indicates rapid growth and a rapidly increasing threat against the trusted number-theoretic cryptosystems Shor challenged, albeit theoretically, in 1997.

With the rising interest in practical implementations of quantum computers, the National Institute of Technology and Standards (NIST, USA) called for new quantum-resistant cryptographic schemes, also called \textit{post-quantum cryptography} or \textit{PQC} for short, to be standardized. This process was announced in 2016 and has up to this point resulted in four submission rounds. In 2022, NIST ended the third submission round by selecting the CRYSTALS-KYBER scheme as a new standard for cryptographic key-establishment algorithms as well as FALCON, CRYSTALS-Dilithium and SPHINCS+ as new standards for digital signatures. Even though four new schemes were selected for standardization, a fourth round continues with remaining schemes and some new ones. 

Generally, NIST plans to standardize schemes based on different mathematical problems, so-called \textit{PQC families}. The four main families based on \textit{coding theory}, \textit{structured lattices}, \textit{systems of multivariate polynomials}, and \textit{hash functions}.
\td{CEHCK}

Multivariate cryptography is the branch of cryptography that revolves around two mathematical problems systems of multivariate polynomials, the MP problem, and the IP problem. The central problem is that of the MP problem \cite{ding2020}, typically specialized to the MQ problem. The MQ problem revolves around solving multivariate \textit{quadratic} polynomials, like 
$$
    p(x_0,x_1,x_2) = x_0x_1 + x_1,
$$
in a system of $m$ polynomials. By ''solve'', the idea is to find the assignments of variables that evaluates $p_i(\mathbf{x}) = 0$, for all $p_i$ in a system $\mathcal{P} = \{p_i(x_{0}, \dots x_{n - 1})\}_{i = 0}^{m - 1}$. Choosing quadratic polynomials over other degrees of polynomials is often a matter of efficiency. The MQ problem is NP-complete, and therefore a balance of efficient implementations but hard to break is important when designing these cryptographic schemes.

In order to sufficiently choose the parameter sizes for these new MQ-based cryptographic schemes, the efficiency of solving such polynomial systems must be taken into account. Using existing and well-known solvers, the NIST candidate scheme \texttt{Rainbow} was broken by Ward Beullens in \cite{crypto-2022-32130}, which shows the necessity for these types of solvers in the cryptanalysis and parameter choice of post quantum cryptographic schemes.

However, the goal of more efficient \textit{MQ-solvers} is not only important for post-quantum cryptography. One umbrella term for different attacks on general cryptographic schemes is that of \textit{algebraic cryptanalysis}. The goal with algebraic cryptanalysis is to map the cipher in question to a system of polynomials (alongside any further necessary information) which, when solved, yields the secret key (in this case for a symmetric key cipher).

Some examples of a reduction from breaking a cipher to the MQ problem are those of \cite{nover2005algebraic}. The first attack, due to Courtois and Pierprzyk \cite{courtois2002cryptanalysis}, models the AES (Rijndael) cipher as a system of multivariate quadratic polynomials over the ring of integers modulo two ($\mathbb{Z}_2$). The second attack, due to Murphy and Robshaw \cite{crypto-2002-1565}, uses the idea of creating a new cipher called \texttt{BES} over an extension field of $\mathbb{Z}_2$. These two attacks do not successfully break AES, however, do potentially yield key extraction methods that are faster than a simple bruteforce procedure.

Other examples of algebraic cryptanalysis are general attacks like \textit{cube attacks} \cite{Videau2011} and importantly also attacks on cryptographic hash functions like that of \cite{fse-2011-23547}. The use-cases for efficient MQ-solvers are therefore vast and they are set to be an important tool in the cryptanalysis of not only PQC schemes, but also more traditional symmetric ciphers and cryptographic hash functions. 

\subsection{Alternative methods for solving multivariate systems over $\eff_2$}
This subsection introduces alternatives to the algorithm focused on in this thesis. The subsection does not does not describe these alternatives in-depth, however, will refer to relevant materials that does so. The section is neither an exhaustive survey of related algorithms.

\paragraph{Gröbner basis solvers.} The concept of a Gröbner basis was introduced by Bruno Buchberger and has been at the core of many algorithms for solving multivariate systems over a finite field. Two important solvers based on the idea of Gröbner bases are $F_4$ and $F_5$ by Jean-Charles Faugère in \cite{FAUGERE199961} and \cite{10.1145/780506.780516}. These algorithms has proven quite useful both in practical implementations and in cryptanalysis. One problem with these algorithms is when applied to overdetermined polynomial systems, i.e. $m > n$, as they typically manipulate pairs of polynomials in a sequential manner. 

\paragraph{XL, eXtended Linearization.} In response to a method called \textit{relinearization}, the authors of \cite{eurocrypt-2000-2187} introduced XL. Although the procedure was shown to have strong theoretical properties, it primarily works well with rather over-defined systems, where $m \gg n$. The general idea of the method is to combine Gröbner bases with large linear systems of equations generated via the input polynomial system. Implementations related to this method of solving multivariate systems have seen some success in the Fukuoka MQ-challenges\multifootnotetag{foot:mq} Type-I and Type-III groups. 

\paragraph{Fast Exhaustive Search.} The FES procedure (fast exhaustive search) is explained in \cref{sec:prereq:fes} and \cref{sec:ext} as it is used internally in Dinur's polynomial-method solver. It was introduced in \cite{ches-2010-23990} and has seen both CPU, GPU, and FPGA implementations. Standalone, this algorithm is a very effective tool for solving systems of multivariate polynomials over $\eff_2$, and has held quite a few speed records for solving these systems.

\paragraph{Crossbred.} In \cite{cryptoeprint:2017/372}, a hybridized algorithm was introduced by Joux and Vitse which was shown, experimentally, to beat previous solve methods. The method uses a combination of manipulations of the input system via a Macaulay matrix representation alongside an exhaustive search on smaller, related, systems. In practice, the procedure holds a fair share of the top 5 performing algorithms in the Fukuoka MQ-challenges\multifootnotetag{foot:mq}.

\multifootnotetagtext[foot:mq]{\url{https://www.mqchallenge.org/}}

\newpage