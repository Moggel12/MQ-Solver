\section{Implementation} \label{sec:impl}
The accompanying git repository contains more than one implementation, or \textit{variant}, of Dinur's original algorithm. These variants are divided into a faster C implementation and a prototype SageMath implementation. C function declarations can be found in the \texttt{inc/} folder, other code can be found under \texttt{src/}.

For alternative implementations of some of the procedures described in \cref{sec:prereq}, see \cite{ches-2010-23990}, \cite{cryptoeprint:2013/436}, and \cite{crypto-2022-32130}.

\subsection{SageMath code} \label{sec:impl:sage}
As was implied earlier, the SageMath implementation of Dinur's algorithm works mostly as a prototype or testing ground for the C implementation. Some optimizations have been tested in this version of the code, prior to it being implemented in C, however, these optimizations worked on an algorithmic level more than on a machine level.  This prototype allowed for approximating the bottleneck areas of the algorithm while essentially also working as a proof-of-concept for using Dinur's algorithm in practice. These approximations of course were rougher in some areas than others, due to the overhead imposed by SageMath and Python.

The prototype implements the three procedures described by Dinur in \cite{eurocrypt-2021-30841}, more or less described as the pseudo-code is presented. The three main procedures described by Dinur can be found in \texttt{src/sage/dinur.sage} with some accompanying convenience and test functions. A bit-sliced version of the FES procedure, described in \cite{ches-2010-23990} and \cref{sec:prereq:fes}, for quadratic polynomials can be found in \texttt{src/sage/fes.sage}. This implementation is not as heavily optimized as those in \cite{ches-2010-23990} and \cite{cryptoeprint:2013/436}, simply due to the SageMath-induced overhead counteracting fine-adjusted optimizations. The prototype code also introduces a FES-based recovery procedure, acting as an alternative to the Möbius Transform originally described by Dinur (see \cref{sec:ext}). The Möbius Transform was implemented in \texttt{src/sage/mob.sage} and allows for a \textit{sparse}-transform used for interpolating the $U$-polynomials. This implementation is rather naive as it interpolates these polynomials \textit{symbolically} using the polynomial classes from SageMath. The choice of switching between FES-based interpolation and using the Möbius transform is a simple boolean switch in the \texttt{solve()} and \texttt{output\_potentials()} functions in \texttt{src/sage/dinur.sage}. \td{IF MOB IS ALTERED IN SAGE; CHANGE THIS} 

The tests for the SageMath procedures can be found in the same file as the procedure they test. This may not be the prettiest setup, but recall that most of this SageMath code is prototype and used for verification of the C-code. As references to practical implementations of these procedures are sparse, the SageMath code was rather important as it eliminated the normal headaches of working in C and allowed for a more theory-near approach. Considering that SageMath has procedures built-in for working with polynomials, matrices and rings, it was a very important stepping stone towards a C implementation. Once the prototype was finished, implementing the C code was much less painful as most of the algorithmic ideas had already been exercised.

Other than the prototype code implemented in SageMath, a \textit{front-end} was also implemented allowing for easier loading, generation, and calling of the optimized C code. For more on this, see \cref{sec:impl:interface}.

\subsubsection{Dinur's core procedures}
\paragraph{SageMath implementation of SOLVE.}
The top-level \texttt{solve} procedure can be found in the \texttt{src/sage/dinur.sage} file. To test it, one may call the \texttt{test\_sage\_solve()} function with appropriate parameters. This implementation of Dinur's algorithm tries to mimic the pseudo-code (see \cref{alg:solve}) closely, e.g. by using dictionaries for checking comparing solutions found in round $k$ with those of earlier rounds. However, by close inspection, one might see that there are few differences between the implementation and the pseudo-code still. In the pseudo-code, Dinur parameterizes the variable $n_1$, allowing variation on how it is chosen. The SageMath implementation fixes this to 
$$
    n_1 \approx \lceil \frac{n}{5.4} \rceil.
$$
The choice of fixing $n_1$ to this specific value stems from Dinur's proof of the time complexity of this algorithm. Setting the parameter to approximately $\frac{n}{5.4}$ ensures that the complexity is balanced between the time evaluating the $U$ polynomials and the time taken for computing the evaluations of $\Tilde{\mathcal{P}}$ in the set $W^{n - n_1}_{w + 1} \times \{0,1\}^{n_1}$. This can be altered in the SageMath source code itself if necessary, however, here it was kept simple.

Another part of the SageMath code that differs from the source material is its 
\texttt{fes\_recovery} parameter. This parameter handles whether or not to use FES-based recovery, described in \cref{sec:ext:fes_interp}, to recover the $U$ polynomials. The parameter is essentially a boolean switch that tells the \texttt{output\_potentials()} function which implementation is needed. A look at the main loop inside the \texttt{solve()} function shows the last \textit{major} deviance from the pseudo-code. Here, instead of allowing the algorithm to run indefinitely the length of the \textit{history} is limited. The limit found here can be changed in \texttt{src/sage/c\_config.py} and defaults to 30.

Generating the matrix $A$ of \cref{alg:solve}, \cref{alg:solve:matrix}, for constructing $\Tilde{\mathcal{P}}$ occurs in \texttt{gen\_matrix\_rank\_l()}. Ensuring that matrix $A$ has rank $\ell$ is a simple Monte Carlo approach generating new matrices until one of the needed rank is acquired. The generation of the matrix makes use of the \texttt{rand()} function from the C standard library. The PRNG is seeded in \texttt{solve()} using the constant \texttt{RSEED}, defaulting to 42. The underlying PRNG may be changed in the \texttt{src/sage/c\_config.py} file as well, however, is useful for simplifying the testing of the C implementation.

Now, the way polynomials are represented in the SageMath code is through the built-in (in SageMath) representation of boolean polynomials. As mentioned earlier, this does incur an overhead but will also simplify certain operations, such as generating the system $\Tilde{\mathcal{P}}$:
\pylisting{src/sage/dinur.sage}{329}{329}
which also eases the process of computing $d_{\Tilde{\mathcal{F}}}$,
\pylisting{src/sage/dinur.sage}{331}{331}
alongside evaluating the polynomials in the system on candidate solutions:
\pylisting{src/sage/dinur.sage}{293}{294}

Finally, instead of going through all $2^{n - n_1}$ assignments for $\hat{y}$, as in \cref{alg:solve}, the SageMath version stores solutions in \texttt{defaultdict}s. This way, the procedure may only iterate through $\hat{y}$ values where $U_0(\hat{y}) = 0$:
\pylisting[label={lst:sage:history}]{src/sage/dinur.sage}{342}{351}
The \texttt{c\_debugging} part should be ignored here. The other parts should then show a strong resemblence to \cref{alg:solve}.

\paragraph{Outputting isolated solutions in reality.} The function \texttt{output\_potentials()} is the SageMath equivalent of \cref{alg:output}. With the purpose of computing isolated solutions using the $U$ polynomials, the SageMath implementation takes two approaches, as noted earlier. The \texttt{fes\_recovery} parameter chooses either a FES-based interpolation and evaluation or the traditional method of using the boolean Möbius transform and using \texttt{compute\_u\_values()}. With the "traditional" method of computing isolated solutions, the code first obtains \texttt{V} and \texttt{ZV}, being a \texttt{defaultdict} and list of \texttt{defaultdict}. Once those have been saved, the procedure goes on to interpolate the $U$ polynomials and store them in an array, \texttt{U}:
\pylisting{src/sage/dinur.sage}{216}{218}
using the appropriate parameters ($w$ for $U_0$ and $w + 1$ for the other $U_i$s). Here, the procedure also includes \texttt{sub\_ring} which essentially is a polynomial ring with indeterminates $x_0$ through $x_{n - n_1 - 1}$ instead of $x_0$ through $x_{n - 1}$, as the $U$ polynomials are defined over $y$ being the first $n_1$ variables. Using this approach helped simplify the prototype implementation, as the Möbius transform then could be implemented by the recursive formula (see \cref{sec:prereq:poly_interp}). This does of course add the overhead of addition and multiplication using SageMath polynomial classes while potentially also using large amounts of stack-based memory. However, as the SageMath code acts as a prototyping platform, this is not required to change.

Although the traditional Möbius transform takes in either an array representing a polynomial to be evaluated or the full set of evaluations in order to interpolate a polynomial, the code called in the snippet above works as discussed originally intended by Dinur. The transform takes in sparse sets of evaluations in order to interpolate the $U_0$ and $U_i$ polynomials, with the weight values defining the recursion depth for the Möbius transform. \td{Finish paragraph}

Note, the way that the procedure is implemented using the Möbius transform for interpolation and evaluation is not the most efficient, however, it does resemble the pseudo-code to quite a large degree. Should one be interested in implementing the Möbius transform in Python or SageMath and running the code with that instead, the code may be extended via the \texttt{src/sage/mob\_new.sage} file. Also, the symbolic method of creating an array representing the polynomial, then evaluating and then converting the output to an array of evaluations (seen in the snippet below)
\pylisting[label={lst:sage:output_mob}]{src/sage/dinur.sage}{228}{251}
should also see a revision if the Möbius transform is more "properly" implemented in \texttt{src/sage/mob\_new.sage}.

\td{DEFINE THE RECURSION SOMEWHERE}

The remainder of this version ensures to convert the evaluations of the $U_i$ polynomials into the actual $z_i$ bit of an isolated solution, depending on the evaluation of $U_0$. The code can be seen below.
\pylisting{src/sage/dinur.sage}{259}{264}
The output dictionary \texttt{out} is stored as a \texttt{defaultdict} essentially due to memory concerns. This of course adds processing, but it should be clear by now that execution speed was not always a priority in the SageMath code.

If the caller alternatively sets the \texttt{fes\_recovery} parameter to \texttt{True}, the algorithm uses the FES-based interpolation and evaluation, described in \cref{sec:ext:fes_interp}. The gist of the code when going with a FES-based interpolation is the following (ignoring the timing code, of course):
\pylisting{src/sage/dinur.sage}{193}{205}
The most notable difference between the two approaches is the combination of interpolation and evaluation into one. Due to the fact that both the $U_0$ and the $U_i$ polynomials are interpolated in the same procedure, the \textit{weight} or \texttt{w} parameter is set to \texttt{w + 1} to accommodate for interpolation of both the $U_i$s and $U_0$. The hybrid approach is described generally in \cref{sec:ext:fes_interp:interp_recover}.

\paragraph{Computing $U$-polynomial interpolation points.} Computing the interpolation points, used for the $U$-polynomials, takes place in much the same way as Dinur described it in \cite{eurocrypt-2021-30841}. The procedure \texttt{compute\_u\_values()} in \texttt{src/sage/dinur.sage} handles this, and is more or less the same setup as \cref{alg:uvalue}. One difference that affects performance is the use of dictionaries (\texttt{defaultdict} specifically), instead of lists, as the storage solution for the interpolation points.
\pylisting{src/sage/dinur.sage}{158}{159} 
As already discussed, regarding \texttt{output\_solutions()}, this choice was merely made due to memory concerns as the \texttt{defaultdict} serves as a way of doing lookups on non-existing keys without taking up too much valuable memory.

Although the \texttt{bruteforce()} procedure is a product of Dinur's use case for the FES procedure, and not necessarily the brainchild of \cite{cryptoeprint:2013/436} and \cite{ches-2010-23990}, it is described in \cref{sec:impl:fes}.

\subsubsection{FES procedures} \label{sec:impl:fes}

\paragraph{Bruteforce and FES as we know it:} One aspect of the \texttt{compute\_u\_values()} process that is different from \cite{eurocrypt-2021-30841} is the \texttt{bruteforce()} procedure described. The description in \cite{eurocrypt-2021-30841} leaves much to the imagination as it is only really stated that the FES procedure of \cite{cryptoeprint:2013/436} and \cite{ches-2010-23990} would be used to evaluate the sparse set of inputs, $W^{n - n_1}_{w + 1} \times \{0,1\}^{n_1}$. However, Dinur does make arguments for two general approaches and their performance penalties/impacts. These alternatives are described as either simply iterating through the set $W^{n - n_1}_{w + 1}$ while going through all $\{0,1\}^{n_1}$ values at each such iteration, or alternatively the use of \textit{monotonic gray codes}. The structure of monotonic gray codes was briefly mentioned in \cref{sec:prereq:fes:gray_codes}. Due to simplicity and the negligible performance penalty, the choice here was to use the former approach. 
\td{Add monotonic gray code description in the section on gray codes}
The \texttt{bruteforce()} procedure can be found in \texttt{src/sage/fes.sage} and is in essence rather simple. The function simply slices the polynomial system into $1 + n + \binom{n}{2}$ integers and uses the bit-sliced representation to actually run the FES procedure on the entire system at once. The procedure loops through a sequence of integers $i = 0, \dots 2^{n - n_1} - 1$ skipping any value of $i$ where the $hw(i) > d$ for a parameter $d = w + 1$. For each value of $i$ where $hw(i) \leq w + 1$, a \textit{prefix} is computed and stored as a list of indices for the $1$-bits in $i$.
\pylisting{src/sage/fes.sage}{173}{173}
This prefix represents the value of $i$, by storing the indices of the 1-bits, and is used for representing the FES procedure \textit{state}.

The FES procedure was not originally intended to evaluate across a sparse set of inputs and so some alterations had to be taken. First off, the FES code relies on a dataclass \texttt{State}:
\pylisting[label={lst:sage:state}]{src/sage/fes.sage}{8}{14}
In many areas, this state representation is similar to that of \cref{alg:fes_eval}, however, here it has added the \texttt{prefix} attribtue. This attribute helps the state class maintain information between different calls to the \texttt{fes\_eval()} procedure. The state, \texttt{s}, is updated whenever a new value of $i \in W^{n - n_1}_{n_1}$ is reached in the counter using the \texttt{update()} procedure:
\pylisting{src/sage/fes.sage}{174}{174}
The way this update process can be thought of is by interpreting the system as being partially evaluated on the bit representation of the counter $i$. Say $i = 5$ and $d = 6$, then $hw(i) = 2 < 6$ and so $i \in W^{n - n_1}_{w + 1}$. Because of this, we essentially partially evaluate the polynomials $p \in \mathcal{P}$ on the first $n - n_1$ variables using the binary representation of $i$. However, instead of going through the process of evaluating the polynomials $p$ in code, the \texttt{state} class acts as a representation of this. When a new counter value, $i' \in W^{n - n_1}_{w + 1}$, is reached the \texttt{update()} function ensures to \textit{turn on} or \textit{turn off} (set variables to 1 or 0, respectively) the bits off in the representation of $d_1$ and $d_2$ (see \cref{sec:prereq:fes} for reference) depending on which bits were turned off and which were turned on going from counter value $i$ to $i'$. An example of this process is the following code:
\pylisting{src/sage/fes.sage}{77}{81}
This snippet goes through all variables that are turned off, going from $i$ to $i'$, and adds the coefficient of the monomial $x_{idx}x_{k + n - n_1}$ to the value of $\frac{\partial f}{\partial x_{k + n - n_1}}$. If it is not clear from the example, adding $n - n_1$ to the indices is due to the fact that we "eliminate" the first $n - n_1$ variable of each $p \in \mathcal{P}$ by assigning them a value from $\{0,1\}$, and so the following FES procedure should only be concerned with searching for solutions for the partially evaluated polynomials (being of $n_1$ variables instead of $n$). Likewise, the "evaluation", or \texttt{s.y}, also takes effect of changing the values of the first $n - n_1$ variables, and so, for this reason, we must ensure to subtract the values of $\frac{\partial f}{\partial x_{idx}}$ from the evaluation \texttt{s.y}.

Similarly, the \texttt{update()} function handles the effect of changing the assignment of some $x_{idx}$ on monomials $x_{idx}x_j$ where $j \leq (n - n_1)$. This process is then proceeded to account for variables in the prefix that are turned \textit{on} and at last assigns the \texttt{state.prefix} the value of \texttt{prefix}. Followingly, the procedure has updated the FES state, and accounted for the effects of changing some variable assignments (of the first $n - n_1$ variables) from $0$ to $1$ or conversely.

The subsequent code in \texttt{bruteforce()} may then simply call the \texttt{fes\_eval()} procedure with the newly update state \texttt{s} as such:
\pylisting{src/sage/fes.sage}{175}{175}
which internally works much like the pseudo-code described in \cref{sec:prereq:fes}. The function \texttt{fes\_eval()} handles two cases, depending on whether or not it is used in tandem with \texttt{bruteforce()} or \texttt{fes\_recover()}. For now, this section will focus on the code of \texttt{bruteforce()} and related parts. The function declaration looks like
\pylisting{src/sage/fes.sage}{118}{118}
where the \texttt{compute\_parity} parameter is set to false when called from \texttt{bruteforce()}, as this then ensures that the procedure returns solutions, like traditionally done in FES. The parameters of \texttt{f}, \texttt{n} and \texttt{n1}, should be clear to anyone that read \cref{sec:prereq}, or \cite{eurocrypt-2021-30841} and \cite{cryptoeprint:2013/436}. the parameters of \texttt{prefix} and \texttt{s} then relate to the state processing described just now.

Whenever the \texttt{fes\_eval()} function is called with \texttt{s = None}, the code has to initiate a new state \texttt{s}. In the same vein as the \texttt{update()} function, the \texttt{init()} function (in \texttt{src/sage/fes.sage}) ensures to initiate the first and second derivatives according to the prefix that the system is being partially evaluated on. This leads to computations like
\pylisting{src/sage/fes.sage}{57}{61}
where the computation is very much like the one in \texttt{update()}. This initialization procedure also accounts for quadratic monomials, where we may assign both variables a value. Of course, other than initializing the state according to the current prefix the state also needs to be initialized in accordance with \cref{alg:fes_init}.

Once the state has been initialized, the execution of \texttt{fes\_eval()} follows the ideas of \cref{alg:fes_eval} closely, again assuming the \texttt{compute\_parities} parameter is set to \texttt{False}. However, the procedure is essentially doing an exhaustive search on the space $\{0,1\}^{n_1}$, but we seek solutions from $W^{n - n_1}_{w + 1} \times \{0, 1\}^{n_1}$. Therefore, instead of storing the value of $s.i \oplus (s.i \rightshift 1)$ (the gray code value of \texttt{s.i}), the binary representation of the prefix has to be prepended to the bitstring as well. 

Once all solutions have been found and stored with the currently set \textit{prefix}, the procedure may not yet return control to \texttt{bruteforce()}. Since the \texttt{state} is re-used and updated between successive calls to \texttt{fes\_eval()}, the state also has to reset certain values. That is, at the end of \texttt{fes\_eval()} the procedure resets \texttt{s.d1}, \texttt{s.d2} and \texttt{s.i} to the values they had at the beginning of the execution of \texttt{fes\_eval()}. The snippet here shows this process:
\pylisting{src/sage/fes.sage}{153}{155}
where it is clear that the code essentially "subtracts" second derivatives added to the first derivative (during the search for solutions) for each of the first $n_1-1$ \textit{unassigned} variables. Following this, the procedure can reset \texttt{s.y}. Inspecting the xor operations of \cref{alg:fes_step}, it should be clear how this process resets \texttt{s.d1} and \texttt{s.y}. Of course, the counter in \texttt{s.i} is also reset to ensure that the next run of \texttt{fes\_eval()} only goes through $\{0,1\}^{n_1}$ as well. 
\td{CONSIDER MORE VISUAL EXAMPLES}
\td{CONSIDER SEPARATING BRUTEFORCE AND FES\_EVAL MORE}
The last part of the \texttt{bruteforce()} procedure in \texttt{src/sage/fes.sage} adds the solutions obtained by \texttt{fes\_eval} as lists of $GF(2)$ elements. Once all prefixes, or all values of $W^{n - n_1}_{w + 1}$, have been processed the algorithm returns all solutions found to the caller.

\paragraph{FES-based recovery:} The procedures STEP, $\text{BIT}_1$, and $\text{BIT}_2$ from \cref{alg:fes_step} are all implemented in a quite straightforward manner, meaning that they will not be explained here. In return, the \texttt{fes\_recover()} function from \texttt{src/sage/fes\_rec.sage} probably deserves some explanation. There are essentially three parts to this function; \texttt{fes\_recover()} itself, \texttt{fes\_eval()} with \texttt{compute\_parities} set to \texttt{True}, and \texttt{part\_eval()}. As the observant reader may have already noticed, the \texttt{fes\_recover()} function is an implementation of the procedure described in \cref{sec:ext:fes_interp} or \cref{alg:fes_recover}. As the idea of combining an interpolation and evaluation procedure, using FES-related means is a rather novel idea it was beneficial to prototype such an implementation in SageMath before attempting a C-version. 

The prototype \texttt{fes\_recover()} procedure acts similarly to the Möbius transform approach of \cref{alg:output}, as it fills an array of size $2^{n - n_1}$ with the evaluations of the $U$ polynomials. Unlike the "traditional" implementation in \texttt{output\_solutions()} of \texttt{src/sage/dinur.sage}, the implementation here bit-slices the $U$ polynomials, such that each of the $2^{n - n_1}$ entries hold $n_1 + 1$ bits in an integer. Due to this, the \texttt{output\_solutions()} function takes this bit-slicing into account as it computes the \texttt{out}-array when the function is called with \texttt{fes\_recovery} set to \texttt{True}.
\td{CHECK DESCRIPTION OF OUTPUT\_SOL AND SEE IF IT MATCHES}
Given its dependence on interpolating the $U$ polynomials via the same set of inputs, $W^{n - n_1}_{w + 1} \times \{0,1\}^{n_1}$, the procedure uses the \texttt{state} class from \texttt{src/sage/fes.sage}, described earlier in this section. Alongside the \texttt{state} class, a prefix also has to be stored, just as was described earlier. This is also where the \texttt{fes\_eval()} flag \texttt{compute\_parities} plays in. Instead of summming the $n_1 + 1$ parities after the call(s) to \texttt{fes\_eval()}, the procedure allows for computing them directly in \texttt{fes\_eval()} for each prefix. Since the interpolation and evaluation are over the $U$ polynomials, which are of higher degree, the \texttt{fes\_eval()} code may not be reused as it only supports quadratic polynomials. Therefore, the procedures described in \cref{alg:fes_step}, \cref{alg:fes_eval}, and \cref{alg:fes_init} are no longer sufficient, as was also described in \cref{sec:ext:fes_interp}. The derivatives now have to be a table instead of independent arrays, as it is not known in advance the degree of $U$ and the level of partial derivatives needed.

Fundamentally, the way the table \texttt{d} stores derivative values of varying degrees is by simply interpreting the counter \texttt{si} as a bitstring, and then selecting maximally \texttt{degree} 1-bits of this bit-string. An example of this is $si = 101101_2$ and $degree = 3$, where the index would be $si = 1101_2 = 13$. This is computed using the \texttt{bits()} function, returning an array of indices for all 1-bits in $si$:
\pylisting{src/sage/fes_rec.sage}{76}{76}
The value \texttt{k} is here simply an array of such indices. This is much the same as $\text{BIT}_1$ and $\text{BIT}_2$ in \cref{alg:fes_step}, just now generally instead. The values in this array \texttt{k} are then used to compute the indices into the derivative table, according to which monomial $si$ represents. An example of the computation of indices into \texttt{d} is:
\pylisting[label={lst:sage:index}]{src/sage/fes_rec.sage}{80}{80}
It should further be noted that index $0$ in \texttt{d} represents the evaluation of \texttt{si} on the $U$ polynomials. 

Then, as the procedure seeks to evaluate the $U$ polynomials on $\{0,1\}^{n - n_1}$, it goes through all values $i = 0, \dots 2^{n - n_1}$. At each iteration, the procedure makes a choice about interpolating or evaluating given the current iteration count $i$. In much the same sense as interpolation using the Möbius transform on sparse inputs, the interpolation of \texttt{fes\_recover} occurs whenever the counter value has $hw(i) < d$ where $d$ is the \textit{degree} specified through the \texttt{degree} parameter to \texttt{fes\_recover}. In all other cases, the procedure conversely evaluates the polynomial, given the interpolations computed.

Focusing on the interpolation part for a bit, it can be seen that the line
\pylisting{src/sage/fes_rec.sage}{91}{91}
is rather prominent, whether it be for interpolating $i = 0$ or $i > 0$. The prefix, much like in \texttt{bruteforce()}, represents the input to the various $U$ polynomials; $U(prefix)$. Internally, the \texttt{part\_eval()} function updates the state \texttt{s} and computes \textit{parities}. These parities are then simply returned alongside the updated state \texttt{s}. This behavior of \texttt{fes\_eval()} is of course invoked by the \texttt{compute\_parities} parameter being set to \texttt{True}. It may be worth noting here that the parities computed in \texttt{fes\_eval()} are computed through the solutions of each $r \in \Tilde{\mathcal{P}_k}$ generated in \cref{alg:solve}. Therefore, even though the \texttt{fes\_recover()} procedure as a whole requires evaluation of degree $d$ polynomials, it still internally uses quadratic polynomials to interpolate and evaluate the $U$ polynomials. It essentially boils down to \texttt{fes\_recover()} having two different kinds of FES in use.

Now, the \textit{parities} are nothing more than the values computed in \texttt{compute\_u\_values()}, however, this time they are computed directly in \texttt{fes\_eval()}. An example of the parity computation in \texttt{fes\_eval()} is 
\pylisting{src/sage/fes.sage}{144}{148}
Again, this is bit-sliced and so each integer contains $n_1 + 1$ bits, corresponding to each of the $U$ polynomials. The snippet above is the \textit{parity computing} version of the main loop of \cref{alg:fes_eval}. Comparing the snippet above to the main loop of \texttt{compute\_u\_values()} (or \cref{alg:uvalue}) it should be clear that these are more-or-less the same computations. Setting \texttt{compute\_parities} in \texttt{fes\_eval()} to \texttt{True} does not affect anything else than how solutions are handled when encountered. This also means that the derivative table \texttt{d} stores derivative values in a bit-sliced format.

Once the \textit{parities} of a certain prefix have been computed, they may be stored in the derivative table immediately as these represent an evaluation of the $U$ polynomials. Once these have been stored, we may do the computations described in \cref{sec:ext:fes_interp} in order to interpolate related derivative table entries and store them for later use. This process is implemented in the following way:
\pylisting[label={lst:sage:ifes_interp}]{src/sage/fes_rec.sage}{97}{109}
Here, the \textit{backtracking} steps for interpolating the derivative table, as described in \cref{sec:ext:fes_interp}, should be clear.
\td{BE SURE TO ADD GOOD DESCRIPTION OF GGCE (GENERAL FES)}

Conversely, if the hamming weight of the counter is sufficiently high, $hw(si) > degree$, we are in a position to evaluate the polynomials instead of interpolating. This process does not need the parity computation, as the low hamming weight entries in the derivative table have already been updated by earlier interpolation steps, as explained in \cref{sec:ext:fes_interp}. Therefore, the code is simply going bottom-up computing the high-order derivatives first, as explained in \cref{sec:ext:fes_interp}. For these computations, the first \texttt{degree} 1-bits of $si$ are once more used in order to compute the derivative table entries required. The following snippet shows these computations:
\pylisting{src/sage/fes_rec.sage}{78}{80}
Examining the snippet above, one may observe the converse nature of the interpolation snippet from earlier in contrast to the snippet here. This should show how evaluation and interpolation are functioning via the same principles.
\td{BE SURE TO EXPLAIN INTERPOLATION AND EVALUATION STUFF IN EXTENSIONS SECTION ALONGSIDE GGCE.}

Once the current iteration either finished interpolation or evaluation, the procedure stores the \texttt{d[0]} value in the array of evaluations:
\pylisting[label={lst:sage:fes_rec_res}]{src/sage/fes_rec.sage}{113}{113}
Given that this approach is a variant of FES, the evaluations stored in \texttt{d[0]} are the result of evaluating the $U$ polynomials on the gray code value of \texttt{si}, and must therefore be stored accordingly in the \texttt{res} array.

This approach of combining the two may help further mitigate memory problems by possibly adding some computational needs, though nothing theory-breaking, while still conforming to the intent of the Möbius transform in \cref{alg:output}. Evaluations and thoughts on this approach can be found in the \cref{sec:impl:c} section as well as \cref{sec:eval}.

\subsubsection{Möbius Transform and utilities}
\paragraph{Brief note on the Möbius transform.} As mentioned earlier, the Möbius transform implementation(s) can be found in \texttt{src/sage/mob\_new.sage}. The procedure \texttt{mob\_transform()} represents the Möbius transform, however, instead of using lists as described in \cref{sec:prereq:poly_interp} the representation uses the built-in SageMath representation of boolean polynomials. This oddity is the reason that \cref{lst:sage:output_mob} is as comprehensive. Implementing the Möbius transform as described in \cref{sec:prereq:poly_interp}, alongside the \textit{sparse} Möbius transform also described, will make much of the code in \cref{lst:sage:output_mob} redundant and more readable. Understanding the \texttt{mob\_transform()} procedure in \texttt{mob\_new.sage} for this reason requires understanding the recursive formula shown in \cref{sec:prereq:poly_interp} as well. In combination with \texttt{\_f\_expand()} in \texttt{mob\_new.sage}, the similarities between the implementation and theory are strong. Implementing the sparse interpolation could then be done by filtering out high-degree monomials after fully interpolating the polynomial from its sparse set of solutions. For any implementations more focused on the original variant of Dinur's solver, implementations like that in \cite{cryptoeprint:2022/1412} should be followed.

\paragraph{Utilities.} The SageMath codebase in \texttt{src/sage/} provide quite a few extra utility functions, depending on the goal. The \texttt{src/sage/utils.sage} file provides procedures for reading and writing Fukuoka MQ-challenge style files\footnote{\url{https://www.mqchallenge.org/}}, bitslicing multilinear quadratic polynomials, generating polynomial systems, fetching C functions from the shared library, and more. These utilities are used throughout the SageMath codebase and also in the \texttt{run.py} script, in order to handle smaller tasks that would not require a dedicated SageMath file. When extending parts of the SageMath code or the \texttt{run.py} script this file should be kept in mind.

\subsection{Core algorithms in C} \label{sec:impl:c}
%\begin{enumerate}
%    \item Actual implementation of Fast Exhaustive Search
%    \subitem Degree-$d$ and quadratic
%    \subitem FES-recover implementation
%    \item Partial evaluation for FES, and why we reuse state
%    \item Dinurs algorithm
%    \item Representation of polynomials in C-code
%    \item Bitslicing
%    \item Getting interpolation points for FES recover
%\end{enumerate}

In many areas, the standard C implementation takes a similar approach to program design as the SageMath counterpart (see \cref{sec:impl:sage}). However, unlike the SageMath intent with the SageMath prototype, the purpose of the C code was to try and push the algorithm and see what kinds of optimizations are beneficial for it. This subsection seeks to describe the general idea behind the C implementation, drawing parallels to the similar parts in the SageMath code and describing the difference alongside their design choices. Evaluating the usefulness of these optimizations is postponed to \cref{sec:eval}. This section focuses on the \textit{shared library} target (see \cref{sec:impl:compile}) as well as the \textit{standardized} implementation, plus any utilities that overlap between codebases.

\subsubsection{Solve, and other top-level procedures} \label{sec:impl:c:solve}
\td{REWRITE EVERYTHING ABOUT POTENTIALSOLUTION STRUCT}
The main entry point into the library is the \texttt{solve()} procedure, residing in \texttt{src/c/standard/mq.c}. This procedure for the most part acts like its SageMath counterpart. However, the \texttt{solve()} procedure expects a bit-sliced system of polynomials in its \texttt{poly\_t *system} parameter. Further, the procedure expects that the monomial ordering is graded lexicographic ordering and that the polynomials in the system have been \textit{linearized}, i.e. the system consists solely of multilinear polynomials. Examining the function declaration, the other parameters should be somewhat self-explanatory:
\clisting{inc/mq.h}{40}{40}
The latter parameter, \texttt{sol}, is an out-parameter containing the solution found, if any.

Now, due to the unrestricted nature of C, the initial part of the procedure allocates memory for the elements like the \textit{sub}-system $\Tilde{\mathcal{P}_k}$, the matrix $A$ and the solution history:
\clisting{src/c/standard/mq.c}{80}{85}
$\Tilde{\mathcal{P}_k}$ and $A$ are both stored in the \texttt{rand\_mat} and \texttt{rand\_sys} variables simply as integer arrays, and reused in each round. The solution history is saved as an array of pointers, called \texttt{potential\_solutions} and its size is defined by the \texttt{MAX\_HISTORY} macro. This macro defines an upper limit on how many rounds the solver may use to search for a solution and can be modified in \texttt{inc/mq\_config.h}. The types of these three variables are not as such default types in C.

One prominent type in both the vectorized and standard implementation is the \texttt{poly\_t} type. This type is simply a type definition reusing different integer types in C:
\clisting{inc/mq_config.h}{90}{90}
The \texttt{POLY\_TYPE} macro is defined as a uint\{8, 16, 32, 64\}\_t for the \textit{standard} implementation, depending on the compile-flags (see \cref{sec:impl:compile}). The definitions of both the \texttt{POLY\_TYPE} macro and \texttt{poly\_t} reside in \texttt{inc/mq\_config.h}.

The other important type is \texttt{PotentialSolutions}. From the naming, some may recognize this as a struct, which would be correct. The definition of this struct can be found in \texttt{inc/fes.h} and is defined as the following:
\clisting[label={lst:c:potsol}]{inc/fes.h}{50}{54}
The reasoning behind defining this struct in \texttt{inc/fes.h}, and why it is used in the first place, is explained later in the section.

Once sufficient memory has been allocated and relevant parameters computed (such as $n_1$ and $\ell$) the main loop of the algorithm starts. In the SageMath version, the matrix generation and subsequent generation of the subsystem $\Tilde{\mathcal{P}_k}$ could be handled by the polynomial and matrix representation built-in. In the C code, these procedures have to be handled manually. 

Generating the $A$ matrices of \cref{alg:solve} is done through a call to \texttt{gen\_matrix()} in \texttt{src/c/utils.c}. The procedure takes as argument an array of \texttt{poly\_t} to fill and the number of rows, \texttt{n\_rows}, and columns, \texttt{n\_columns}, for the matrix. Since the matrix is supposed to consist of elements from $\eff_2$, each row is an integer sampled in the \texttt{gen\_row()} procedure. These integers are masked so that only the bottom \texttt{n\_cloumns} bits are left. The \texttt{gen\_matrix()} procedure then simply generates the full $\ell \times m$ matrix and computes the rank, by computing the row echelon form of the matrix and counting independent rows in the meantime. The procedure continues to generate matrices until one of rank $\ell$ is obtained.

Given the matrix generated consists of $m$ bits inside a \texttt{poly\_t}, the procedure for generating $\Tilde{\mathcal{P}_k}$ is quite straightforward. Recall that the new system is generated by
\begin{equation} \label{eq:tilde_p_k}
    \Tilde{\mathcal{P}_k} = \left\{\sum_{j = 0}^{m - 1} A_{i,j} \cdot p_j(\mathbf{x})\right\}_{i=0}^{\ell - 1}
\end{equation}
The \texttt{compute\_e\_k()} procedure (named after $\Tilde{E}_k$ from \cite{eurocrypt-2021-30841} which in this thesis is named $\Tilde{\mathcal{P}_k}$ for consistency with other parts) generates the new system polynomial-by-polynomial, term-by-term. Now, this procedure generates a system of multilinear polynomials, keeping it consistent with the rest of the codebase. As the polynomials are saved bit-sliced, the summation and multiplication in \cref{eq:tilde_p_k} is merely a matter of:
\clisting{src/c/standard/mq.c}{34}{34}
Here, \texttt{GF2\_ADD} is a macro for bitwise \textit{xor} and \texttt{GF2\_MUL} is a macro for bitwise \textit{and}, these and other macros are described in \cref{sec:impl:c_abstr}. The variable \texttt{s} above is equivalent to the $i$ in \cref{eq:tilde_p_k}. The snippet above is the computation for the constant terms of $\Tilde{\mathcal{P}_k}$, however, by sufficient indexing into \texttt{old\_sys} and \texttt{new\_sys} the same approach may be taken to compute other terms as well. At last, the procedure also computes the degrees of these polynomials meanwhile.

Once a new system of polynomials has been computed, the procedure goes on to compute the potential solutions for the current iteration. This is where one of the larger deviations from the SageMath code and \cref{alg:solve} takes place. First, the C solver solely makes use of FES-based recovery (see \cref{sec:ext:fes_interp}) and the Möbius transform was not implemented. Second, the \texttt{solve()} procedure calls the \texttt{fes\_recover()} procedure directly:
\clisting{src/c/standard/mq.c}{118}{118}

Since the procedure of \cref{alg:output} primarily combines interpolation, evaluation, and "translating" evaluations into potential solutions, the benefit of actually implementing it in combination with the \texttt{fes\_recover()} procedure was not clear. With the structure of \texttt{fes\_recover()}, the evaluations can be translated and stored directly in the array of solutions as they are computed, instead of going through $2^{n - n_1}$ iterations to (in \texttt{fes\_recover()}) to compute all evaluations to then followingly iterate through all $2^{n - n_1}$ evaluations once more in order to translate them into potential solutions (\cref{alg:output:recover} in \cref{alg:output}). If a Möbius transform procedure would have been implemented, the necessity for an implementation of \cref{alg:output} may have been larger. The actual implementation of \texttt{fes\_recover()} is discussed in \cref{sec:impl:c:fes} and \cref{sec:impl:opt}.

Since the FES-based recovery of \cref{sec:ext:fes_interp} is equivalent to the Möbius transforms (interpolation \textit{and} evaluation) of \cref{alg:output}, breaking the theory is not a risk. Now, the reasoning behind not including a Möbius transform implementation was the \textit{implementability}, so to say. The \texttt{fes\_recover()} version was the more desirable choice, both in order to introduce a use-case for this new procedure but also due to the seemingly simple nature of it. This, however, does not mean the Möbius transform cannot be implemented nor that \texttt{fes\_recover()} should be the sole choice for any other implementations of Dinur's polynomial method solver.

The potential solutions computed in \texttt{fes\_recover()} are stored in the struct \texttt{PotentialSolution}. In this representation, the $\mathbf{y}$- and $\mathbf{z}$-bits of the potential solution are stored in their separate integers. Further, the $\mathbf{y}$-bits are \textit{not} stored as gray codes.

With the \textit{solution checking} phase of the C implementation, the code once again diverges from the SageMath and reference (\cref{alg:solve}) material. Instead of storing the solutions in a dictionary or full $2^{n - n_1} \times (n_1 + 1)$ size array, only potential solutions with $U_0(\mathbf{y}) = 1$ are stored sequentially in an array. The storage of these lists of potential solutions is done through the struct:
\clisting{src/c/standard/mq.c}{18}{22}
Since the code cannot simply make a lookup into a dictionary or array using the $\mathbf{y}$-bits, the history checking phase differs from \cref{alg:solve} and \cref{lst:sage:history}. 

Due to the way \texttt{fes\_recover()} is implemented, the \texttt{PotentialSolution}s are stored in with the \texttt{y\_idx} values in increasing order. Since the conversion to and from Gray code ordering acts like a bijective mapping, two \texttt{y\_idx} values that are equal correspondingly mean that their Gray code values are equal. Knowing this, the procedure may simply loop through all previous lists of potential solutions, using the sorting of the \texttt{y\_idx} values as and indicator of whether or not a solution could still be found.
\clisting{src/c/standard/mq.c}{156}{156}
Here, each iteration then goes through the \texttt{PotentialSolution}s stored in the history:
\clisting{src/c/standard/mq.c}{158}{159}
until it encounters a historic solution with a larger \texttt{y\_idx} value; 
\clisting{src/c/standard/mq.c}{164}{167}
or one where \texttt{y\_idx} and \texttt{z\_bits} are identical across the historic solution and the current one;
\clisting{src/c/standard/mq.c}{168}{169}
This ensures that the procedure checks only the strictly necessary historic solutions linearly. Alternatively, a dictionary could of course be implemented, however, it is unclear whether or not this would prove a dramatic benefit. Another alternative would be to allocate all $2^{n - n_1} \times (n_1 + 1)$ entries in a table as originally proposed in \cite{eurocrypt-2021-30841}. Of course, choosing to allocate room for all evaluations of the $U$ polynomials will drastically limit the problem sizes that the algorithm may handle in practice (on machines without unlimited memory).

\td{CHECK THAT GRAY CODE ORDERING IS MENTIONED IN SAGEMATH FES\_RECOVER}

Finally, should the procedure encounter identical \texttt{y\_idx} and \texttt{z\_bits} values across two potential solutions, the code acts more or less as expected:
\clisting[label={lst:c:history_check}]{src/c/standard/mq.c}{171}{195}
In this snippet, the code ensures to convert the \texttt{y\_idx} value into its Gray code value, before combining the $\mathbf{y}$- and $\mathbf{z}$-bits into one integer (or \textit{potential solution}). The potential solution is then evaluated on the system to check if it is an actual solution to the system $\mathcal{P}$. If so, the procedure cleans up and returns with the solution stored in the out-parameter \texttt{poly\_t *sol}.

If no solution was found, or an error occurred along the way, the procedure returns $1$. Therefore, if a $1$ is returned the value in the \texttt{sol} pointer is an invalid solution. Returning a zero conversely means success and that \texttt{sol} contains a valid solution.

\subsubsection{FES} \label{sec:impl:c:fes}
\paragraph{FES-recovery in C.} Examining the \texttt{src/c/standard/fes.c} file many of the procedures have a corresponding implementation in the SageMath code. Procedures like \texttt{init()}, \texttt{update()}, \texttt{step()}, and the various bit-indexing functions are implemented by the same principles in both the C codebases and the SageMath codebase. Therefore, procedures like these will not be discussed greatly in this section.

Since the implementations of \cref{alg:solve} in C both solely make use of the \texttt{fes\_recover()} approach from \cref{sec:ext}, this section will mainly describe the approach taken for the non-vectorized version of \texttt{fes\_recover()}. Finally, at the end of the section a simple FES implementation, as described in \cref{sec:prereq:fes}, will be described.

All FES related procedures and declarations can be found in \texttt{inc/fes.h} and \texttt{src/c/standard/fes.c} (or \texttt{src/c/vectorized/fes.c}, for the vectorized version). Examining the declarations in the header file for non-vectorized compilation targets, the most important declarations are 
\clisting[label={lst:c:state}]{inc/fes.h}{58}{65} % C State
being the state used by FES to store relevant values like derivatives and the prefix. Similarities to \cref{lst:sage:state} should be rather clear. The \texttt{uint8\_t *prefix} acts as an array in much the same way as the \texttt{prefix} attribute in \cref{lst:sage:state}, although integers are limited to a single byte. This is more than sufficient, given that the solver could not feasibly solve systems with around $256$ variables, and so the positions saved in \texttt{uint8\_t *prefix} would never reach such amounts.

Next is the \texttt{fes\_recover()} declaration. A look into the file would also show declarations for \texttt{fes()} and \texttt{bruteforce()}, however, as mentioned these procedures are explained near the end of this subsection. The \texttt{fes\_recover()} declaration is quite similar to that of its SageMath counterpart
\clisting{inc/fes.h}{112}{114} % C fes_recover declaration
The use of the \texttt{PotentialSolution} struct for storing solutions was already described, and replaces the need for \texttt{res} array used in the SageMath implementation of \texttt{fes\_recover()} (\cref{lst:sage:fes_rec_res}). The return value is made to be an error code, whereas \texttt{PotentialSolution *results} and \texttt{size\_t *res\_size} are out-parameters for the potential solutions found and the amount of these, respectively.

Inside \texttt{src/c/standard/fes.c}, the two top-most functions are \texttt{init\_state()} and \texttt{destroy\_state()}. These act like a constructor and a destructor, respectively, for the \texttt{state} struct shown in \cref{lst:c:state}. The constructor allocates heap memory according to parameters of the problem instance, like $n$ and $n_1$. Consequently, the memory has to be freed to avoid memory leaks, which is handled in \texttt{destroy\_state()}. For anyone seeking to expand the codebase, the \texttt{init\_state()} procedure will not take ownership over any passed pointers but only copy their data.

Next, in the same file, the \texttt{fes\_recover()} procedure is found. Although the similarities between the C implementation and \cref{alg:fes_recover} (or the SageMath version in \texttt{src/sage/fes\_rec.sage}) are noticable, some areas still deserve explanation. In the state of the procedure, the code initializes the different tables and structures needed for a run-through
\clisting{src/c/standard/fes.c}{454}{470}
The most interesting would be the derivative table \texttt{poyl\_t *d} allocated at \srcref{src:c:d_alloc}. For the same reason as not storing potential solutions in a dictionary, the derivatives are not stored in one either. Instead, the table is a dynamically allocated array and the actual size of the array is stored in \texttt{size\_t d\_size}. The size is calculated as 
\begin{equation}
    d_{size} = \sum_{i = 0}^{d} \binom{n - n_1}{i}.
\end{equation}
The indexing itself is computed using the \texttt{monomial\_to\_index()} function (from \texttt{src/c/standard/fes.c}), in order to work with the size of the \texttt{d} table:
\clisting[label={lst:c:index}]{src/c/standard/fes.c}{94}{112}
The reasoning behind this different indexing scheme can be found in \cref{sec:impl:opt:fes_rec}.

Once all tables and values have been setup, the procedure starts computing the \textit{parities}, or evaluations of the $U$ polynomials. Like in the SageMath code, this is done using a procedure called \texttt{part\_eval()}, handling \texttt{state} updates and calls to \texttt{fes\_eval\_parity()}. Instead of combining this alternative FES version, where parities are computed instead of storing solutions, with the one used in Dinur's \texttt{BRUTEFORCE()} procedure. In the \textit{standard} C codebase, the \texttt{fes\_eval\_solutions()} computes and stores solutions to the input system, while \texttt{fes\_eval\_parities()} computes the parities from \cref{alg:uvalue}.

Once parities have been computed and returned to the \texttt{fes\_recover()} procedure, it directly checks the parities for a potential solution and does relevant conversions if needed. Recall that the C codebase mitigates having a procedure like \cref{alg:output}, and therefore instead requires solution checking and conversion somewhere else, being directly in \texttt{fes\_recover()} in this case. The code for checking solutions is 
\clisting[label={lst:c:solution_check}]{src/c/standard/fes.c}{482}{487}
The above snippet shows how parities are handled for $\hat{y} = 0$, whereas a similar snippet can be found for $0 < \hat{y} \leq 2^{n - n_1}$. Beware, the macros used in the conditional check are explained in \cref{sec:impl:c_abstr}. The conditional checks for the case where $U_0$ evaluates to one (recall \cref{sec:dinur:dinur_alg}), as the parities and $\mathbf{y}$-bits then form a potential solution to the system $\mathcal{P}$. If so, the \texttt{PotentialSolution} struct is used to store the \textit{counter} value as is (not its Gray code value) and inverts the evaluations of the $U_i$s before storing them as \texttt{z\_bits}.

As already mentioned, much of the procedure functions like \cref{alg:fes_recover} and the SageMath counterpart. Therefore, the hybrid approach is mostly the same. Computing the bit-positions required (as \cref{alg:fes_recover:bits} from \cref{alg:fes_recover}) is done through the \texttt{bits()} function, als located in \texttt{src/c/standard/fes.c}. The positions are stored in an array \texttt{k}, like the SageMath counterpart does it 
\clisting{src/c/standard/fes.c}{500}{500}
which may then be used to compute indices using \texttt{monomial\_to\_index()} (\cref{lst:c:index}), as follows
\clisting{src/c/standard/fes.c}{504}{505}

In the interpolation phase, the $\mathbf{y}$-bits are represented by an array of the bit-positions for each 1-bit in the counter-value (limited to the first $n - n_1$ bits). The storage solution for this prefix agrees with the \texttt{state} struct (see \cref{lst:c:state}) as it is used for this purpose. Computing the prefix, the Gray code value of the counter \texttt{si} is used:
\clisting{src/c/standard/fes.c}{520}{523}

Now, the rest of the interpolation phase is setup much like \cref{lst:sage:ifes_interp}, and will therefore not be explained in-depth. Due to the nuances of C, when compared to SageMath, there are certain less interesting differences. See \cref{sec:impl:fes} for an explanation of the SageMath code, \cref{alg:fes_recover} for the general approach, and \texttt{src/c/standard/fes.c} for the source code.

In each iteration of the outer-most loop of \texttt{fes\_recover()}, having either evaluated or interpolated, the procedure checks \texttt{d[0]} for a potential solution. Recall that \texttt{d[0]} stores the evaluations of all the $U$ polynomials in a bit-sliced configuration. Checking and the following processing of the potential solution is more-or-less the same as \cref{lst:c:solution_check}, however, this time the \texttt{y\_idx} entry stores the value of the counter \texttt{si}. Also, instead of checking directly against the \texttt{parities} variable, it checks \texttt{d[0]} as both the interpolation and evaluation cases will store evaluations of the $U$ polynomials in $d[0]$.

Should an error occur during execution of \texttt{fes\_recover()}, the procedure cleans up its memory and returns 1, much like the error codes in the \texttt{solve()} procedure. In case no error occurred, the procedure returns 0.

\paragraph{FES for the curious.} A simple implementation of the FES approach (for quadratic polynomials) discussed in \cref{sec:prereq:fes} was also implemented. This exists mostly for comparison against Dinur's solver, however, it may still be used for solving systems. The remaining part of this subsection is dedicated to a brief look into this.

There are two methods for running FES directly in the C code; the \texttt{bruteforce()} procedure and \texttt{fes()} procedure. The \texttt{bruteforce()} procedure is an implementation of the \texttt{BRUTEFORCE()} procedure from \cref{alg:uvalue}. The declaration can be found in \texttt{inc/fes.h} and the implementation in \texttt{src/c/standard/fes.c}. The procedure is declared as 
\clisting{inc/fes.h}{82}{83}
which is rather similar to that of \cref{alg:uvalue}. Here, the \texttt{poly\_t *solutions} parameter is an out-parameter for storing solutions. The return value is the number of solutions stored.

If the goal is to run FES more in line with \cite{cryptoeprint:2013/436} or \cite{ches-2010-23990}, the \texttt{fes()} function is defined to be able to handle this:
\clisting{inc/fes.h}{96}{96}
This procedure is not as heavily optimized as the GPU version of \cite{ches-2010-23990}, or the FPGA version of \cite{cryptoeprint:2013/436}, but has received the same level of \textit{care} as the procedure \texttt{fes\_eval()} used inside \texttt{fes\_recover()}. The solutions found are stored in the parameter \texttt{poly\_t *solutions} and the return value holds the amount found. An added \textit{bonus} for this procedure is the ability to read benchmarks directly, if needed. More on benchmarking in \cref{sec:impl:c:util_bench}

\subsubsection{Utilities and benchmarking} \label{sec:impl:c:util_bench}
Besides the C procedure discussed up to this point, a few more can be found in \texttt{src/c/standard/utils.c} and \texttt{src/c/standard/benchmark.c}. The most notable utility procedures are the \texttt{gen\_matrix()} procedure and the \texttt{eval()} procedure. These were briefly described in \cref{sec:impl:c:solve}. Declarations for these procedures can be found in \texttt{inc/utils.h}, if needed for extending or reusing the codebase.

When the shared library is compiled, a few benchmarking tools are also available. First off, if the goal is to solely benchmark the solving procedure, the procedure 
\clisting{inc/benchmark.h}{33}{33}
takes as input a list of bit-sliced systems that will then be run through the \texttt{solve()} function. The procedure computes both the time of different points of interest in the codebase, but also computes the balance of interpolation versus evaluation in \texttt{fes\_recover()}. All of the computed values are averaged according to how many calls to \texttt{solve()} that finished successfully, i.e. returned a valid solution. The stored timings and iteration counts can be accessed through the global variables declared in \texttt{inc/benchmark.h}, prefixed with a \texttt{g\_}. 

Benchmark timings are computed using the \texttt{BEGIN\_BENCH()} and \texttt{END\_BENCH()} function like macros, as well as \texttt{READ\_BENCH()}. The names should make the macros somewhat self-explanatory. Should there be interest in adding more benchmarks, declaring an extern variable in \texttt{inc/benchmark.h} for storing timings (or another type of benchmark value), initializing it to a suitable value in a user-chosen position, and then adding appropriate macro-calls at the points of interest.
\td{ADD NOTE ABOUT FES BENCHMARKING}
\subsection{Optimizations} \label{sec:impl:opt}
% \begin{enumerate}
%     \item Tight integration of U-value computation with polynomial interpolation and full evaluation
%     \item Sparse Möbius transform
%     \item FES-recover
%     \item FES-recover derivative table
%     \item C-specific optimizations
%     \item Handling memory
%     \item Concurrency
% \end{enumerate}
Different kinds of optimizations were implemented either directly in the codebase described in \cref{sec:impl:c} or in the alternative codebase in \texttt{src/c/vectorized/}. This subsection seeks to go through the most prominent optimizations and why they are used.

\subsubsection{FES-based recovery} \label{sec:impl:opt:fes_rec}
\paragraph{Internal optimizations to FES-recover.} For the C codebases, the choice of procedures for interpolating and evaluating the $U$ polynomials was \texttt{fes\_recover()}. One benefit of going with \texttt{fes\_recover()} is that interpolation and evaluation of these polynomials could be done in \textit{one} pass-through of their $2^{n - n_1}$ possible inputs, instead of two separate calls to the Möbius transform.

Another effect of using the \texttt{fes\_recover()} approach was the ability to more easily store only the potential solutions. Alternatively, as is described in \cref{sec:dinur:dinur_alg}, the procedure would have to store all 
$$
    2^{n - n_1}
$$ 
evaluations. In \cite{eurocrypt-2021-30841} it was shown that the number of suggested solutions in a round was about 
$$
    2^{n - 2n_1},
$$
and so the saved memory could yield beneficial for larger values of $n_1$. Although the current implementation does allocate a large chunk of memory at once, simply adding some memory reallocation code. Other alternatives for benefitting from compactly storing potential solutions also exist, depending on what the end-goal is.

Another memory-saving element used in \texttt{fes\_recover()} is the alternative indexing scheme used in the C code, compared to that of the SageMath code. This alternative indexing was described in \cref{sec:impl:c:fes}. Reusing the SageMath code, the derivative table $d$ would have to be exponential in size, as monomial representations would be sparsely scattered depending on their bit-string representation. As already mentioned, the approach taken instead yields a memory consumption of 
$$
    \sum_{i = 0 }^{d} \binom{n - n_1}{i} < \left(\frac{(n - n_1)e}{d}\right)^d
$$
where $d$ is the degree passed to \texttt{fes\_recover()} as \texttt{deg}. Comparing the \textit{polynomial} memory usage to an exponential one where entries are sparsely scattered, the benefits should be quite clear. Of course, the disadvantage of this approach is the additional computational overhead used to compute the different indices, compared to the more \textit{naive} approach chosen in the SageMath prototype (\cref{sec:impl:fes}).
\td{NOT POLYNOMIAL}
In total, the memory consumption of \texttt{fes\_recover()} is lowered, and in expectation, it is rather competitive with using in-place Möbius transforms like \cref{alg:mob}. Using the memory-efficient Möbius transform suggested in \cite{eurocrypt-2021-30841}, could pose a strong alternative to this approach.
\td{CACHE?}

\paragraph*{Removing \texttt{output\_potentials}} Although the SageMath prototype kept \texttt{output\_potentials()}, the choice was to leave it out in the C code. Due to the choice of using the evaluation and interpolation process of \cref{sec:ext:fes_interp}, integrating solution "filtering" and processing directly into the procedure was more intuitive.

Since each iteration of the outer-most loop of \cref{alg:fes_recover} (and thereby its implementations) discretely either evaluate or interpolate, the process of filtering and processing potential solutions directly in this loop was easily integrated. Other than allowing for a tighter compacting of potential solutions in the returned array, this process also removes the need for the processing at \cref{alg:output:recover} in \cref{alg:output}. Therefore, the time taken to go through all solutions once more, in order to \textit{post-process}, is no longer effecting the total solve time. Of course, as already mentioned, this comes at the cost of potentially having more computational load at each iteration of \texttt{fes\_recover()}.

\subsubsection{Vectorization and parallelization} \label{sec:impl:opt:parallel}
Various methods of parallelizing the procedure were used, including the use of SIMD instructions and multicore CPUs. Of course, the shared library is compiled with GCCs \texttt{-O3} flag, which implies quite a few optimizations occur at compile-time. However, some optimizations can be hard to determine statically and so active care must be taken when developing. 

A smaller method of optimizing storage is the use of the different compile flags available. The compile flags can currently tell the solver what size of integer the input systems and their solutions should be stored in. This can help reduce the overall memory footprint. Currently, the SIMD implementation divides $\mathcal{P}$ and $\Tilde{\mathcal{P}_k}$ into separate integer sizes, which is not readily available in the non-SIMD version. Using smaller integer sizes for the smaller system $\Tilde{\mathcal{P}}$ allows for storing more polynomial terms in a cacheline, boosting performance on smaller levels. As these terms are accessed quite frequently, and in an almost sequential manner, this can boost performance on a low level.

Although it is possible to vary integer sizes for both the standard and vectorized implementations, both are limited to input systems of 64 polynomials and 64 variables (maximally). This may be extended, see \cref{sec:impl:c_abstr}.

\paragraph{Bit-slicing.} As bit-slicing is a simple method of obtaining linear speedups, especially when used in the context of systems of boolean functions, it was a safe choice in this case. Since Dinur's polynomial-method algorithm works with multiple polynomials at once in almost every area, many of the procedures and sub-procedures benefit from this.

In both the non-SIMD builds and the vectorized builds, the input system to \texttt{solve()} should be bit-sliced (mentioned in \cref{sec:impl:c:solve}). The "sub"-system $\Tilde{\mathcal{P}}$ is generated in \texttt{compute\_e\_k()} (both the version in \texttt{src/c/standard/mq.c} and \texttt{src/c/vectorized/mq.c}) with this in mind. The multiplication of individual matrix indices, recall \cref{eq:tilde_p_k}, with terms of the polynomials of $\mathcal{P}$ is quite easily implemented with bit-wise operations. This later part of course assumes that both $\mathcal{P}$ and the matrix $A$ are bit-sliced, which they would be in this case. Then the summation part is simply a matter of computing the parity of the integer representing said multiplication. 

Also in \texttt{fes\_recover()}, bit-slicing benefits the internal calls to \texttt{fes\_eval\_parity()} as well as the evaluations of the $n_1 + 1$ polynomials; $U$. The evaluations of these polynomials, i.e. the parities computed in \texttt{fes\_eval\_parity()} and in the evaluation phase of \cref{alg:fes_recover}, are stored bit-sliced as well. Processing and checking for candidates is then merely a matter of bit-masking. 

At last, evaluating candidate solutions on the system $P$ also benefits from this. By computing term-by-term the evaluation of said term against the relevant variable assignments, this process may be parallelized by representing a zero-assignment as an integer of $m$ 0s, and a one-assignment as an integer of $m$ 1s.
\td{ADD PARTIAL EVAL AND BITSLICING SECTION}

\paragraph{Multicore.} Another optimization strategy is the use of multicore CPUs. In order to solve larger systems, a good solver would make use of many of the resources available. Through the \texttt{run.py} script, the solver can be spawned in a multicore context, wherein each core runs a separate \textit{instance} of the input system.

Say the CPU(s) of the system has (have) a combined $q$ cores available. Running 
$$
    n_{partial} = \lfloor \log_2 q \rfloor
$$ 
solvers in parallel is possible by \textit{partially evaluating} the input system $\mathcal{P}$. With $n_{partial}$ variables, $2^{n_{partial}}$ variable assignments are possible. Therefore, by fixing $n_{partial}$ variables, each of the $q$ cores (assuming $q$ is a power of two of course) can get its own partially evaluated system, i.e. an independent input system. As no state is shared between these $2^{n_{partial}}$ partially evaluated systems, they are ideal candidates for a multi-core system. Once a core found a solution, the variable assignment of the fixed variables would then be appropriately added to the solution.

Currently, the multicore operation is implemented directly in the \texttt{run.py} script, using Pythons built-in \texttt{multiprocess} library. Using this library for concurrency ensures that no problem with Python's infamous Global Interpreter Lock is met. However, the variables for the input system are fixed in Python, handing each fixed system to its own process that then internally calls the \texttt{solve()} function from the shared library (either vectorized or standard, depending on what was compiled).

\td{MAKE NICE FIGURE}

\paragraph{SIMD instructions.} As the optimization target is x86-based CPUs, the use of SIMD instructions is possible. In \cite{eurocrypt-2021-30841}, the optimization strategy used alongside the memory-efficient Möbius transform was to pre-compute multiple $\Tilde{\mathcal{P}_k}$ systems and interleave their respective evaluations of the $U$ polynomials with solution testing. This approach is mimicked with the use of SIMD instructions and \texttt{fes\_recover()}.
\td{DEFINITELY NEEDS A REVISION}
Using appropriate compile-flags (see \cref{sec:impl:compile}) the \texttt{mq.so} library may either include AVX or AVX2-based instructions (specifically, C intrinsics). Now, using the vector register available in modern x86 CPUs, either 128 bits or 256 bits are available in a single register. Since the \texttt{solve()} function internally sets 
$$
    n_1 = \lceil \frac{n}{5.4}\rceil,
$$
and the $\Tilde{\mathcal{P}_k}$ systems consist of $\ell = n_1 + 1$ polynomials, an input system of $m \leq 37$ would yield that $\Tilde{\mathcal{P}_k}$ fits into a single byte. For systems $\mathcal{P}$ with $m \leq 37$, it is possible to pack either $\frac{128}{8} = 16$, or $ \frac{256}{8} = 32$, byte-sized systems $\Tilde{\mathcal{P}_k}$ into a single AVX register. Likewise, for $\mathcal{P}$ with $37 < m \leq 64$, an AVX register can hold either $\frac{128}{16} = 8$ or $\frac{256}{16} = 16$ systems $\Tilde{\mathcal{P}_k}$ in separate 16-bit integers. 

Given that the expected number of rounds to find a solution is 4 (shown in \cite{eurocrypt-2021-30841}), the vectorized version of the \texttt{solve()} function uses the AVX registers as follows: Say an AVX register can hold up to $j$ systems $\Tilde{\mathcal{P}_k}$. By fixing 
$$
    \log_2 \frac{j}{4}
$$ 
variables of $\mathcal{P}$, each partially evaluated version of $\mathcal{P}$ gets its own four vector elements, for the $\Tilde{\mathcal{P}_k}$ systems. Then, each round of \texttt{solve()} computes four $\Tilde{\mathcal{P}_k}$ systems for each fixed version of $\mathcal{P}$ and stores them all in one AVX register. This way the procedure is allowed to operate in a lock-step manner on multiple systems at once, without having to synchronize between threads or processes. 

For the most part, the procedure follows the likes of the \textit{standard} approach described in \cref{sec:impl:c}, however, some parts may need explanation. In the setup of \texttt{solve()} in \texttt{src/c/vectorized/mq\_vectorized.c}, the procedure fixes the input system $\mathcal{P}$ and stores the fixed systems in new and smaller integer arrays
\clisting{src/c/vectorized/mq_vectorized.c}{134}{139}
which can be seen in the snippet above. The variable macro \texttt{FIXED\_VARS} is set depending on the compilation flags (see \cref{sec:impl:compile}), but represents the number of variables fixed for the current setup. The variable \texttt{new\_n} is set to be 
\clisting{src/c/vectorized/mq_vectorized.c}{128}{128}
which then is used internally instead of the original \texttt{n} variable for most computations.

The procedure \texttt{gen\_matrix()} is kept the same (\texttt{src/c/utils.c}), primarily to comply with the testing framework. However, instead of computing only one matrix in each round of \texttt{solve()}, the procedure computes one for each index in the vector register. The \texttt{rand\_sys} variable, traditionally representing $\Tilde{\mathcal{P}_k}$, is a bit different though. Instead of \texttt{poly\_t *rand\_sys}, the generated systems are stored as \texttt{poly\_vec\_t *}:
\clisting{src/c/vectorized/mq_vectorized.c}{145}{145}
Also noticed the \texttt{aligned\_alloc} call. An effect of using AVX registers is that in any memory location storing vector values, the memory should be correctly aligned. This could be mitigated by using explicit using instructions for unaligned memory but will serve a performance penalty.

The \texttt{fes\_recover()} is also different for the vectorized versions. The function declaration can still be found in \texttt{inc/fes.h}, but it looks a bit different:
\clisting{inc/fes.h}{44}{46} 
The procedure takes both the original system \texttt{poly\_t *system} (non-fixed) as well as the vector of $\Tilde{\mathcal{P}_k}$ systems. Also, instead of expecting a sufficiently sized array in \texttt{poly\_t *result}, the procedure uses the parameter to return an actual solution.

Going into \texttt{src/c/vectorized/fes.c}, much of the code may seem similar to that explained in \cref{sec:impl:c}. Much of the logic is on an abstract level the same, but instead of using \texttt{poly\_t *} to represent systems, most procedures instead use \texttt{poly\_vec\_t *}. The various function-like macros prepended with \texttt{VEC\_} represent an abstraction layer over the AVX intrinsics used to work with AVX registers. Recall that each AVX register contains four rounds worth of systems, for evaluations of $\mathcal{P}$ on some fixture. Therefore, lines like 
\clisting[label={lst:c:vector_parity}]{src/c/vectorized/fes_vectorized.c}{364}{370}
display how the aforementioned lock-stepping works in practice. The non-vectorized version of the above snippet is
\clisting{src/c/standard/fes.c}{353}{357}
for which the differences between the vectorized and non-vectorized versions show. Both snippets are from \texttt{fes\_eval\_parity()} in the vectorized and non-vectorized versions, respectively. Since the vectorized version runs multiple systems in parallel during the execution, conditionals have to be handled differently. E.g. for each system in the vector \texttt{s->y}, in \cref{lst:c:vector_parity}, a combination of AVX masks and blends are used to mimic multiple parallel conditional checks at once. These principles are the same throughout \texttt{src/c/vectorized/fes\_vectorized.c}.

At last, having shown how much of the vectorized codebase is similar to the standard codebase, the biggest difference between the two codebases is how solutions, checking and processing, are handled. The snippet 
\clisting{src/c/vectorized/fes_vectorized.c}{448}{477}
shows how the vectorized codebase handles evaluations of $U$ polynomials. As an AVX register contains multiple $\Tilde{\mathcal{P}_k}$ systems it computes the evaluations of the $U$ polynomials in lock-step as well. Therefore, for each evaluation of the $U$ polynomials, the system may simply check if overlapping values exist in the groups of four vector elements in the register. If an overlap is found, the procedure may extract all overlapping solutions directly from the register, construct the solution and evaluate on $\mathcal{P}$ directly. If a solution is found, the procedure may return. If no full evaluation on $\mathcal{P}$ succeeds, the procedure returns an error and starts over in a new iteration in \texttt{solve()}.

Therefore, because evaluations of $U$ the polynomials are computed and checked simultaneously, no large lists of solutions are stored. This will drastically affect memory consumption and entirely mitigates the need for storing solutions. The cost of this is that the procedure potentially skips somre solutions as each round of \texttt{solve()} in \texttt{src/c/vectorized/mq\_vectorized.c} logically corresponds to computing four rounds in the \textit{standard} version and throwing away all potential solutions found if none was found in the last four rounds. 

\subsection{C abstractions} \label{sec:impl:c_abstr}
The C codebases contain certain levels of abstractions, in order to be more extendable. One example of this is the use of macros in the \textit{standard} codebase. Recall from previous subsections that many places in the C code, macro-like functions such as \texttt{INT\_IS\_ZERO}, \texttt{INT\_IDX}, or \texttt{GF2\_MUL} are called (\cref{lst:c:index}, \cref{lst:c:history_check} to give some example listings). These macros are used as an abstractional layer in order to hide the structure of the underlying \texttt{poly\_t} type. If the goal is to extend the codebase to support wider integer types, allowing for larger systems bit-sliced into an integer, these macros help hide the operands used. This could be used in case the wider integer type does not support certain operations. E.g., GCC has support for 128-bit integers on target machines with wide enough integer modes, however, these do have limitations in certain areas and may therefore require workarounds that can be hidden using aforementioned macros. The macros defined for integers can be found in \texttt{inc/mq\_uni.h}.

Likewise, the vectorized codebase hides many operations behind macros for the same reason. These macros can be found in \texttt{inc/mq\_vec.h} which internally makes use of either \texttt{inc/vec128\_config.h} or \texttt{inc/vec256\_config.h}. The latter two header-files also include relevant macros for stating how many variables to fix given the provided compile flags, the amount of elements in a vector register (given the integer sizes stored in it), etc. Say the goal is to extend the codebase to use AVX512 as well, or future AVX (maybe even Intels AMX instructions). The process is a matter of creating a file similar to \texttt{inc/vec128\_config.h} or \texttt{inc/vec256\_config.h}, adding an appropriate conditional case in \texttt{inc/mq\_vec.h}, and finally running the code. Exactly which macros need to be present in the newly added vector-config header can be read near the top of \texttt{inc/mq\_vec.h}. 

Depending on how exotic the vector instruction set is, it may be necessary to add or change functionality in \texttt{src/c/vectorized/vector\_utils.h} as well. If it is simply a matter of adding support for a newer version of AVX, then the approach just described should be fine.

Ensuring that macros like \texttt{FIXED\_VARS} and \texttt{VECTOR\_ELEMENTS} are set correctly, ensures that the vectorized codebase is automatically set up, at compile time, for fixing the correct amount of variables and only storing four rounds for each fixed polynomial.

\subsection{Testing the code}
The following subsection contains some brief notes on the testing methodology and how it was executed. For more on how to run and test different areas of the codebase, see the \texttt{README.md} file in the accompanying repository.

\subsubsection{SageMath}

For each of the procedures implemented in Python or SageMath, the accompanying test functions can be found in the same file as the procedure being tested. I.e. tests related to \texttt{bruteforce()} and \texttt{fes\_eval()} can be found in \texttt{src/sage/fes.sage}, while tests related to \texttt{solve()}, \texttt{output\_potentials} and \texttt{compute\_u\_values()} are found in \texttt{src/sage/dinur.sage}.

Since this SageMath prototype would eventually act as a reference point for the C implementation, the testing approach was to essentially create unit tests for select parts of the codebase. For procedures such as \texttt{output\_potentials}, the testing methodology was to evaluate the $U$ polynomials in their entirety. I.e.
\pylisting{src/sage/dinur.sage}{118}{118}
is the computation of the sums related to the $U_i$ polynomials ($i = 1, \dots n_1$). This testing strategy, where values for the theoretical constructs are computed directly and compared against the outputs of their target procedures, is repeated whenever possible. This way, the testing process is not bound to only run with test cases where answers are known in advance. In addition, the testing framework allows for generating multiple systems and storing any input system that would fail in a test, so it may be reused later.

These tests may be run using the \texttt{run.py} script with the \texttt{-t} flag, specifying what test should be run. To list all available tests, use the \texttt{-l} flag. Information about the \texttt{run.py} script can be found in the accompanying \texttt{README.md} file and partly in \cref{sec:impl:interface}.

\subsubsection{C code}

For the C implementation, there are different levels of tests. Either one may compile the \texttt{bin/test} binary, using the appropriate compilation flags and targets (see \cref{sec:impl:compile}, or alternatively one may run tests using the shared library \texttt{bin/mq.so} (if compiled). In both cases, the results of the C implementations are compared against relevant reference points in the SageMath code implementations. This way, going from the unit tests for the SageMath code, we may compare directly the C code to the verified SageMath code. The test functions for the C code may also be run via the \texttt{run.py} script.

Running tests with the \texttt{bin/test} executable file memory sanitizers, and debugging information are enabled. These tests are used for testing procedures like the computation of the $\Tilde{P}_k$ \textit{sub}-systems and comparing it against the SageMath version. In general, these tests are fed relevant inputs via the SageMath code as well as the desired result(s) and checks against them internally. All these tests may be found in \texttt{src/sage/dinur.sage} and contain a postfix of \texttt{\_SAN} in their function name.

The alternative is to test different functions using the \texttt{bin/mq.so} shared library. Other than testing for correctness, these tests also help test the bridge between the shared library and Python/SageMath. Like the tests for the SageMath code itself, these tests may be found in the related \texttt{.sage} files; tests for \texttt{fes()} and \texttt{bruteforce()} may be found in \texttt{fes.sage}, \texttt{fes\_recover()} in \texttt{src/sage/fes\_rec.sage}, etc.

It could be argued here that more tests should exist, especially for the C codebase. Many new procedures were introduced in both the \textit{standard} and \textit{vectorized} codebases, implying that further testing of these individual parts could have taken place on a more granular level.

\subsection{Interface for running the C code} \label{sec:impl:interface}
The C code was developed to be usable via multiple means. Once a compiled shared library, \texttt{bin/mq.so}, is available the code can either be loaded into other projects that support \texttt{.so} files, or it can be run as a complete solution via \texttt{run.py}. 

Running the solver as a script, an invocation without any flags results in a non-parallelized solve routine using the version of the build that was specified at compile time (in essence; \textit{standard} or \textit{vectorized}). The invocation will either ask for a path to an MQ-challenge\footnote{\url{https://www.mqchallenge.org/}} style text file from which the system may be loaded, or for relevant parameters in order to generate the systems before solving. Giving the \texttt{-p} or \texttt{--parallelize} flag when invoking \texttt{run.py} allows the script to parallelize the solver by fixing variables according to how many CPU cores are present on the system. This last part was also described in \cref{sec:impl:opt}.

Using the shared library on its own in an independent project is also quite straightforward. Although quite a few header files are present in the \texttt{inc/} folder, the most important one is \texttt{inc/mq.h}. If instead the goal is to use other parts of the codebase one may include \texttt{inc/fes.h} for FES-related declarations, \texttt{inc/utils.h} for different utilities like evaluating polynomials, generating matrices or simply computing indices for the bit-sliced representation, or \texttt{inc/vector\_utils.h} for utilities for the vectorized implementations. Definitions and macros used throughout the codebase may be included via the \texttt{inc/mq\_config.h} file. At last, the benchmarking code can also be called on the shared library file. This provides the ability to hook in and read benchmarks directly in C code, and of course, call the benchmark procedure as a whole. Relevant benchmark declarations are stored in \texttt{inc/benchmark.h}. As the benchmark values are simply stored as global variables, it is not recommended to do much more than read the variables. 

Note, common for any function that takes a \texttt{poly\_t} array (or pointer) as input is that the stored system is expected to be bit-sliced, linearized, and (monomials) stored in \textit{graded lexicographic order}.

The C code can of course also be executed via the \texttt{bin/test} executable, which may be compiled with the accompanying Makefile. However, this executable is made to primarily work in tandem with tests in the SageMath code, therefore it may not be of great interest to execute the solver this way.

\subsection{Compilation and compile-time parameters} \label{sec:impl:compile}
The accompanying Makefile can conform to multiple platforms using either vectorized instructions or building for machines with different register sizes. Building any of these different targets requires altering the \texttt{BTIS} flag when calling the \texttt{make}. Setting \texttt{BITS} to 8, 16, 32, or 64 means building the non-AVX optimized version, but regulates the integer sizes used to store the polynomials and solutions to the given width. Specifying 128 or 256 means that the build uses 128-bit or 256-bit registers, respectively. 

The default target creates the file \texttt{bin/mq.so}, ready for dynamic linking into other projects, with \texttt{-O3} optimization. This shared object is described further in \cref{sec:impl:c}. Running \texttt{make tests} will create an executable \texttt{bin/test} with memory sanitizers and debug flags enabled. 
\td{128 AND 256 NOT SUPPORTED IN TEST FILE YET}

While compiling the target, the makefile ensures that a few files are generated as well. One file is \texttt{src/sage/.compile\_config}, ensuring better interoperability between the SageMath code and the C code. A more prominent file is the \texttt{inc/binom.h} file generated. This is essentially a header file containing a lookup table \textit{of sufficient size} alongside the necessary macros to do lookups with a few calculations. This lookup table is generated by the Makefile which internally calls the \texttt{binom.py} script and saves the output in \texttt{inc/binom.h}.
\td{INCLUDE RUN\_TEST.PY?}

\newpage